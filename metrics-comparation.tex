\section{Метрики сравнения облаков точек} \label{section:metrics-comparation}

\subsection{Проблема сравнения облаков точек}
% УВЕЛИЧЕНИЕ ОБЪЕМА!! SSIMD

Работа с облаками точек привносит ряд проблем в алгоритмы для их обработки и принципиально отличается от работы с двухмерными изображнениями. К примеру, в отличиии от двухмерных изображений, облака точек не имеют жесткой \textit{решетчатой структуры (grid-like structure)}, что приводит к невозможности использования многих алгоритмов из области обработки двухмерных ихображений при решении данной проблемы. Другой, не менее важной проблемой, является проблема упорядочивания самих точек в облаке. В общем случае, никакого порядка точек не существует, что создает инвариант перстановки (permutation invariant): изменение порядка точек в облаке приводит к другому облаку, описывающему абсолютно такой же трехмерный объект.

Вышеперечисленные проблемы сильно усложняют процесс сравнения двух облаков точек, который необходим для вычисления потерь алгоритма при восстановлении формы объектов, и приводят к необходимости выработки метрики, инвариантной к перестановкам порядка точек в облаке.


\subsection{Метрики сравнения облаков точек}

Одной из наиболее известных метрик для вычисления расстояния между компактными подмножествами метрического пространства является \textit{метрика Хаусдорфа}:

\[
	d_{H}(X,\;Y)=\max \left\{\sup\limits_{{x\in X}}\inf\limits_{{y\in Y}}|xy|,\;\sup\limits_{{y\in Y}}\inf\limits_{{x\in X}}|xy|\right\}
\]

Однако она является крайне неусточивой к небольшим выбросам во множествах.

% УВЕЛИЧЕНИЕ ОБЪЕМА!! привести пример
% http://dfgm.math.msu.su/files/ivanov-tuzhilin/2014-2015/METRGEOM2014-1.pdf
% привести пример

В работах \cite{metrics-source}, \cite{lrgm-cloud} были предложены две метрики: \textit{CD расстояние, (Расстояние Чамфера, Chamfer distance, CD)} и \textit{EMD расстояние (Earth Mover’s distance, EMD)}.

\subsubsection{CD расстояние}

Пусть \(S_{1}, S_{2} \subset \R^3\), тогда CD расстояние между ними определяется как:

\[
	d_{CD}(S_{1}, S_{2}) = \sum\limits_{x \in S_{1}} \min\limits_{y \in S_{2}} \norm{x - y}_{2}^{2} + \sum\limits_{x \in S_{1}} \min\limits_{y \in S_{2}} \norm{x - y}_{2}^{2}
\]

\medskip
Строго говоря CD расстояние не является метрикой, так-как не выполняется неравенство треугольника. Вычисление данной метрики для каждой пары точек проихсодит независимо, в связи с чем данная задача может быть эффективно распараллелена и скорость вычисления данной функции кратно увеличится. Также операции поиска ближайшего соседа могут быть существенно ускорены путем использования различных пространственных структура данных, таких как \textit{KD-деревья (KD-tree)}. 

% УВЕЛИЧЕНИЕ ОБЪЕМА!! В самом деле метрика + картинка
% УВЕЛИЧЕНИЕ ОБЪЕМА!! OpenCL
\subsubsection{EMD расстояние}

Рассмотрим множества \(S_{1}, S_{2} \subset \R^3\) одинакового размера \(s = \abs{S_{1}} = \abs{S_{2}}\). Тогда EMD расстояние между ними определяется как:

\[
	d_{EMD}(S_{1}, S_{2}) = \min\limits_{\phi:S_{1} \mapsto S_{2}} \sum\limits_{x \in S_{1}} \norm{x - \phi(x)}_{2}
\] где отображение \(\phi:S_{1} \mapsto S_{2}\) является биекцией.


Стоит отметить, что нахождение EMD расстояния аналогично решению задачи о назначениях \cite{assignment-task-1}, \cite{assignment-task-2}, которая решается \textit{Венгерским алгоритмом}\cite{hungarian-alg} за полиномиальное время (в худшем случае $O(n^4)$). Разумеется, на практике, точное вычисление EMD будет крайне ресурсозатратным, особенно для таких объектов как трехмерные изображения. В связи с чем, авторы статьи \cite{metrics-source} предлагают использовать аппроксимацию, которая позволяет более быстро вычислять ответ, пусть и с небольшими неточностями. Необходимо отметить, что данный метод работает для облаков точек с одинаковыми количествами точек.

% УВЕЛИЧЕНИЕ ОБЪЕМА!! Ухудшение из-за биекции
\bigskip
В результате, данные метрики обладают следующими свойствами:
\begin{enumerate}
\item Диффернцируемость относительно координат точек сравниваемых множеств
\item Устойчивость к малым выборсам
\item Относительная вычислительная эффективность для CD расстояния и для EMD аппроксимации
\end{enumerate}



В данной работе, при сравнении облаков точек, мы будем использовать CD расстояние, как более простое, более эффективное в реализации, интуитивно понятное и умеющее работать с облаками точек, состоящих из разных количеств точек.

