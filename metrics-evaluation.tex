\section{Метрики для оценки генеративных моделей для трехмерных данных} \label{section:metrics-evaluation}

Имея метод для сравненения двух облаков точек (см гл. \ref{section:metrics-comparation}) можно перейти к созданию метрики, оценивающей потери алгоритма при воспроизведении форм исходных объектов.

Пусть у нас имеется набор облаков точек \(A\) и набор облаков точек \(B\). Требуется оценить насколько хорошо набор \(A\) воспроизводит форму исходных данных, представленных набором \(B\).

В работе \cite{lrgm-cloud} авторы предлагают ряд метрик для данной задачи. Опишем две из них: \textit{Покрытие (Coverage, COV)} и \textit{Минимальное сопоставимое расстояние (Minimum matching distance, MMD)}.


\subsection{COV метрика}

Для каждого объекта из набора \(A\), находим ближайшего соседа к нему из набора \(B\). Покрытие (COV метрика) считается как отношение количества объектов из набора \(B\), которые были сопоставлены объектам из набора \(A\), к числу всех объектов в наборе \(B\). Чем больше значение данной метрики, тем лучше набора \(A\) покрывает набор \(B\). \cite{lrgm-cloud}

% that were matched
\subsection{MMD метрика}

Для вычисления данной метрики, мы каждому объекту из набора \(B\) сопоставляем ближайший к нему объект из набора \(A\) и затем высчитываем среднюю велечину расстояний. Чем меньше значение данной метрики, тем ближе набора \(A\) лежат к из набору \(B\) \cite{lrgm-cloud}


\subsection{Обсуждение}

В итоге получается, что можно говорить о том, что набора \(A\) хорошо покарывает...

COV метрика показывает насколько хорошо происходит покрытие набора \(B\) объектами из набора \(B\), а MMD метрика показывает точностсь (близость) набора \(A\) к набору \(B\). 
Несложно придумать случай, когда COV метрика будет иметь большое значение 
В связи с этим COV и MMD метрики друг-друга отично дополняют.