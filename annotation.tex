\noindent\textbf{Синтез трехмерных моделей методами машинного обучения}
\newline
Дмитрий Иванов 
\newline
Синтез трехмерных моделей является важной задачей в области машинного обучения и имеет ряд практических применений. Одними из самых важных из них является синтез искуственных данных. Искусственные данные позволяют увеличивать обучающие выборки и увеличивать вариабельность выборок, делая таким образом алгоритмы машинного обучения более надежными. Несмотря на последний прогресс в области разработки генеративных моделей (GAN, VAE), данные методы сложно применить для облаков точек в силу отсутствия в них жесткой решетчатой структуры и небольших размеров выборок данных. В рамках данной работы был разработан метод для генерации трехмерных облаков точек, основанный на методах ICP и PCA. Для сравнения и оценки генерируемых данных были использованы специализированные для работы с облаками точек метрики: Champher distance, Coverage metric и Minimum matching distance metric. Полученный в результате алгоритм просто релизуется, имеет высокую скорость работы, хорошую точность и может быть применим для синтеза искуственных данных и изучения статистических свойств форм объектов.

\bigskip

\noindent\textbf{3D (Three-Dimensional) model synthesis with machine learning methods}
\newline
Dmitry Ivanov
\newline
The problem of 3D model synthesis by machine learning methods is important and has several practical applications.
The generation of artificial data is one of the most important applications of this. Artificial data allows one to increase the dataset's size and data variability. This yields more robust machine learning algorithms. In recent years there has been great progress in the field of generative models. Methods such as GAN and VAE were introduced. Unfortunately, these methods do not apply for 3D point clouds due to their non-grid structure and the small size of these kind of datasets. This paper introduces a new method of generating 3D point clouds. This method is based on ICP and PCA algorithms. New metrics such as the Champher Distance, Coverage and the Minimum Matching Distance were introduced for comparison and evaluation of generated 3D point clouds. The suggested algorithm is actually rather simple, yet it offers good precision and a high level of performance. It can be used for the synthesis of artificial data and for statistical shape analysis.