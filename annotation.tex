\textbf{Синтез трехмерных моделей методами машинного обучения}
\newline
Синтез трехмерных моделей является важной задачей в области машинного обучения и имеет ряд практических применений. Одними из самых важных из них является синтез искуственных данных. Искусственные данные позволяют увеличивать обучающие выборки и увеличивать вариабельность выборок, делая таким образом алгоритмы машинного обучения более надежными. Несмотря на последний прогресс в области разработки генеративных моделей (GAN, VAE), данные методы сложно применить для облаков точек в силу отсутствия в них жесткой решетчатой структуры и небольших размеров выборок данных. В рамках данной работы был разработан метод для генерации трехмерных облаков точек, основанный на методах ICP и PCA. Для сравнения и оценки генерируемых данных были использованы специализированные для работы с облаками точек метрики: Champher distance, Coverage metric и Minimum matching distance metric. Полученный в результате алгоритм просто релизуется, имеет высокую скорость работы, хорошую точность и может быть применим для синтеза искуственных данных и изучения статистических свойств форм объектов.


\textbf{3d models synthesis by machine learning methods}
\newline

The problem of 3d models synthesis by machine learning methods is important and has several practical applications.
The generation of artificial data is one of the most important applications of this. Atrificial data allow increasing the dataset size and data variability. It yields to more robust machine learning algorithms.
In last years there were a big progress in the field of generative models. Such mehods as GAN and VAE were introduced. Unfortunately, these methods don't apply for 3d point clouds due to its non-grid structure and small sizes of datasets. In this work the new method for the generation of 3d point clouds was suggested. This method is based on ICP and PCA algorithms. New metrics such as Champher distance, Coverage metric and Minimum matching distance metric was introduced for comparation and evaluation generated 3d point clouds. The suggested algorithm is rather simple in realization and has good precision and high performance. It can be used for the syntesis of artificial data and for statisitical shape analysis.