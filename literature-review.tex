\section{Обзор существующих методов}

\subsection{Методы основанные на глубинном обучении}

Данная область активно развивается и в ней можно выделить два основных подхода: подход, основанный на работе с вокселями (трехмерными пикселями), и подход, основанный на работе с облаками точек. \par
Воксельный подход, использующий для генерации новых моделей идею генеративно-состязательных сетей, описан в работе \cite{3d-gan}. Аналогичный подход использующий для генерации новых объектов идею вариационного автоэнкодировщика, описан в работе \cite{3d-autoencoder}.\par
Стоит отметиться, что воксельный подход из-за своей низкой разрешающей способности показа неудовлетворительные результаты.

Методы, основанные на работе с облаками точек и использующие различные формы вариационного автоэнкодировщика, описаны в работах \cite{lrgm-cloud}, \cite{adversarial-autoencoder}. Данные методы, несмотря на более высокое качество, по сравнению с воксельным подходом требуют больших объемов обучающих выборок и используют парадигму нейросетевых вычислений.

\subsection{Методы основанные на статистическом анализе формы объектов}

В данной области существует несколько хороших обзоров методов \cite{stat-shape-1}, \cite{stat-shape-2}.
Во всех данных работах для выравнивания объектов используется алгоритм Global Procrustes \cite{procrustes-1}, \cite{procrustes-2}, так-как в тех случаях уже существует взаимно однозначное соответствие между точками различных объектов, которое обычно получается с помощью предварительного поиска 
\textit{ключевых точек (feature points, landmark points)} на объектах. К примеру для работы с лицом выделяются ключевые точки лица: левый уголок левого глаза, правый уголоко левого глаза, нос, уголки рта и т.д. И зная, назначение каждой полученной точки, можно поставить ей в соответствтие такие же точки из других объектов. 
Сам метод, основанный на PCA и предложенный в обзорах для изучения статистических свойств объектов и генерации новых объектов, показал хорошую эффективность и был использован в данной работе. \par
Для решения проблемы сопоставления точек были изучены различные версии алгоритма ICP \cite{icp-main}, \cite{icp-2} и его приложения для пролемы выравнивания объектов \cite{non-rigid-icp}. На основе данных методов был реализован алгоритм решающий проблему взаимно однозначного соответствия точек для трехмерных объектов.

\subsection{Методы сравнения облаков точек и оценки восстанавливающей способности алгоритмов}

В работах \cite{lrgm-cloud}, \cite{metrics-source} были предложены и подробно описаны специализированные метрики для сравнения облаков точек. К примеру, метрика CD (Champher distance) представляют из себя расширенную на трехмерный случай классическую метрику сравнения изображений. \par
Для сравнения двух наборов облаков точек и оценки степени покрытия одним набором другого, были предложены две специализированные метрики и показана необходимость их совместного использования. Стоит отметить, что вычисление данных метрик ресурсоемко, но, при этом, возможно распараллеливание.