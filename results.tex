\section{Результаты} \label{section:results}

В целом, алгоритм показал себя хорошо: на некоторах типов зубов показатели COV метрики достигали значений 0.9. Что при столь малом размере выборки является очень хорошим результатом. Для избеганий зависимости от разбиений выборки на тестовую и обучающую части была использована кросс-валиадация, с коэффициентом разбиения 10.

\subsection{Программная реализаций}

Алгоритм был реализован на языке Python3 с использованием библиотек Numpy, Open3D и Sklearn. Для построения графиков использовалась библиотека matplotlib. В качестве среды разработки использовался Jupyter Notebook.

Для обучения на вход алгоритму подается обучающая выборка. Для генерации объектов алгоритму необходимо передать число используемых главных компонент и количество генерируемых данных.

Алгоритм обучения и генерации выборки размером в 150 элементов на компьютере MacBook Pro(2015) с процессором 2,7 GHz Intel Core i5 при размере обучающей выборки в 150 элементов работает меньше чем за секунду. Во многом, такая скорость была достигнута за счет использования KD-деревьев для расчета CD-расстояния (см. \ref{section:metrics-comparation}).


Тем не менее, расчет метрик для оценивания качества сгенерированных объектов занимает существенно большое количество времени. К примеру, вычисление COV и MMD метрик между двумя группами объектов в 150 и 15 элементов и при среднем количестве точек в объекте равном 2500 точек занимает 28 секунд. 


\subsection{Метрики}

\subsection{Зависимость от $\sigma$}


\subsection{Примеры данных}
На рисунках представлены результаты генерации новых зубов.